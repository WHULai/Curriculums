\documentclass[12pt]{article}
%Also I made it 12pt

\usepackage[fontset=macnew]{ctex}
\usepackage{physics}
\usepackage{tikz}
\usetikzlibrary{3d,calc,patterns,chains,arrows.meta, positioning}
% \usepackage{tkz-euclide}
\usepackage{amsmath}
\usepackage{upgreek}
\usepackage{amsthm}
\usepackage{amsfonts}
\input{PomLingHandoutTemplatePreamble}
%MJKD note to future self - this preamble is just the section headers + PomLing formatting, but an ordering paradox between the two files made me combine them and re-order fontspec. *shrug* In future if it needs an update, just take the PomLing formatting file and add in the section headers for handouts.

\newcommand{\rmd}{\mathrm{d}}
\newcommand{\deriv}[2]{\frac{\rmd #1}{\rmd #2}}
\newcommand{\pderiv}[2]{\frac{\partial #1}{\partial #2}}
\newcommand{\dpderiv}[2]{\dfrac{\partial #1}{\partial #2}}
\newcommand{\dderiv}[2]{\dfrac{\rmd #1}{\rmd #2}}

\title{磁路和变压器}
\author{\href{mailto:lai-wei@whu.edu.cn}{Lai Wei}}
\date{\today}

\begin{document}

\maketitle

\section{磁路及其分析方法}

\subsection{磁场的基本物理量}

磁感应强度由洛伦兹力,即载流导体在磁场受力定义,即

\begin{equation}
    B = \frac{F}{q \cdot v} \quad
\end{equation}

磁通

\begin{equation}
    \Phi = \oint \mathbf{B} \rmd \mathbf{S}
\end{equation}

在均匀磁场中

\begin{equation}
    \Phi = BS \quad \text{或} \quad B = \frac{\Phi}{S}
\end{equation}

根据电磁感应定律的公式

\begin{equation}
    e = -N \deriv{\Phi}{t}
\end{equation}

由此可知磁通的单位是伏秒(V \(\cdot\) s),通常称为韦[伯](Wb)

磁场强度\(H\)是计算磁场时所引用的一个物理量,也是矢量,单位是安每米(A/m)。

磁导率\(\mu\)是一个用来表示磁场媒质磁性的物理量。

\begin{equation}
    B = \mu H
\end{equation}

磁导率的单位是亨[特]每米(H/m)。

任意一种物质的磁导率\(\mu\)和真空的磁导率\(\mu_0\)的比值,称为该物质的相对磁导率\(\mu_{\text{r}}\),即

\begin{equation}
    \mu_{\text{r}} = \frac{\mu}{\mu_0}
\end{equation}

\subsection{磁路的分析方法}

根据安培环路定律

\begin{equation}
    \oint H \rmd l = \sum I
\end{equation}

可得出

\begin{equation}
    H l = N I
\end{equation}

其中\(N\)是线圈的匝数,\(l\)是磁路(闭合曲线)的平均长度,\(H\)是磁路铁心的磁场强度

磁通势

\begin{equation}
    F = NI
\end{equation}

又

\begin{equation}
    NI = Hl = \frac{B}{\mu}l = \frac{\Phi}{\mu S}l
\end{equation}

于是磁路的欧姆定律

\begin{equation}
    \Phi = \frac{NI}{\dfrac{l}{\mu S}} = \frac{F}{R_m}
\end{equation}

其中\(R_m\)是磁路的磁阻,\(S\)为磁路的截面积。(式中\(\mu\)不是常数,所以该式只能用于定性分析,不能用于定量计算)

如果磁路是由不同材料或不同长度和截面积的几段组成的,则用下式计算

\begin{equation}
    NI = H_1l_1 + H_2l_2 + \cdots = \Sigma (HI)
\end{equation}

\(H_1l_1, H_2l_2, \cdots\)也常称为磁路各段的磁压降。

\subsubsection{总结}

\begin{center}
\begin{tabular}{l|l}
  磁路 & 电路  \\
  \midrule
  磁通势\(F\) & 电动势\(E\) \\
  磁通\(\Phi\)  & 电流\(I\)  \\
  磁感应强度\(B\) & 电流密度\(J\) \\
  磁阻\(R_m = \dfrac{l}{\mu S}\) & 电阻\(R = \dfrac{l}{\gamma S}\) \\
  \\
  \(\Phi = \dfrac{F}{R_m} = \dfrac{NI}{\dfrac{l}{\mu S}}\)& \(I = \dfrac{E}{R} = \dfrac{E}{\dfrac{l}{\gamma S}}\) \\
\end{tabular}
\end{center}

\section{交流铁心线圈电路}

\subsection{电磁关系}

磁通势\textit{Ni}产生的磁通绝大部份通过铁心而闭合,这部分磁通称为主磁通或工作磁通\(\Phi\),产生主磁电动势\(e\)。
漏磁通\(\Phi_\sigma\)产生漏磁电动势\(e_\sigma\)

设主磁通\(\Phi = \Phi_m \sin \omega t\),则

\begin{equation}
    \begin{aligned}
        e &= -N \deriv{\Phi}{t} = -N \omega \Phi_m \cos \omega t \\
        &=2 \uppi f N \Phi_m \sin (\omega t - 90^\circ) = E_m \sin (\omega t - 90^\circ)
    \end{aligned}
    \label{11}
\end{equation}

漏磁电动势

\begin{equation}
    e_\sigma = -N \deriv{\Phi_\sigma}{t} = - L_\sigma\deriv{i}{t}
\end{equation}

(漏磁电感\(L_\sigma = \frac{N \Phi}{i}\)为常数)

式\ref{11}中\(E_m = 2 \uppi f N \Phi_m\),是主磁电动势\(e\)的幅值,而其有效值则为

\begin{equation}
    E = \frac{E_m}{\sqrt{2}} = \frac{2 \uppi f N \Phi_m}{\sqrt{2}} = 4.44 f N \Phi_m
\end{equation}

而电源电压(\(R\)和\(X_\sigma\)较小,其压降与\(E\)相比可忽略。)

\begin{equation}
    U \approx E = 4.44 f N \Phi_m = 4.44 f N B_m S
    \label{16}
\end{equation}

\subsection{功率损耗}

线圈电阻上的损耗\textbf{铜损耗}

\begin{equation}
\Delta P_{\text{Cu}} = R I^2
\end{equation}

交流变化下的铁心损耗\textbf{铁损耗}

\begin{equation}
\Delta P_{\text{Fe}} \propto B_m^2
\end{equation}

\begin{enumerate}
\item 磁滞损耗$\Delta P_h$:选用磁滞回线狭小的磁性材料制造铁心(如硅钢)。
\item 涡流损耗$\Delta P_e$:为减少涡流损耗,应选用彼此绝缘的硅钢片叠成铁心,限制涡流,使涡流只能在很小的截面内流通(如图\ref{涡流损耗}(b)所示)。或采用电阻率高的铁心(如硅钢片)。
\end{enumerate}

\begin{figure}[!h]
\centering
\includegraphics[width = .3\textwidth]{graphics/Screenshot 2025-09-11 at 21.07.39.png}
\caption{涡流损耗}
\label{涡流损耗}
\end{figure}

由上可知,铁心线圈交流电路的有功功率为

\begin{equation}
P = UI \cos \varphi = RI^2 + \Delta P_{\text{Fe}}
\end{equation}

\section{变压器}

\subsection{变压器的工作原理}

\subsubsection{电压变换}

电阻压降和漏磁电动势较小,与主磁电动势比较起来,可以忽略不计。于是

\begin{equation}
    u_1 \approx - e_1 \quad \dot{U_1} \approx - \dot{E_1}
\end{equation}

根据式\ref{16},\(e_1\)的有效值为

\begin{equation}
    E_1 = 4.44 f N_1 \Phi_m \approx U_1
\end{equation}

同理,对二次绕组电路

\begin{equation}
    e_2 = R_2 i_2 + (-e_{\sigma 2}) + u_2
\end{equation}

感应电动势\(e_2\)的有效值为

\begin{equation}
    E_2 = 4.44 f N_2 \Phi_m
\end{equation}

变压器空载时

\begin{equation}
    I_2=0, E_2 = U_{20}
\end{equation}

式中\(U_{20}\)式空载时二次绕组的端电压。

\begin{figure}[!h]
    \centering
    \includegraphics[width = .4\textwidth]{graphics/Screenshot 2025-09-13 at 00.57.19.png}
    \caption{变压器的原理图}
    \label{变压器的原理图}
\end{figure}

一次、二次绕组的电压之比为

\begin{equation}
    \frac{U_1}{U_2} \approx \frac{E_1}{E_2} = \frac{N_1}{N_2} = K
    \label{25}
\end{equation}

式中\(K\)称为变压器的\textbf{变比},亦即一次、二次绕组的匝数比。

\subsubsection{电流变换}

由\(U_1 \approx E_1 = 4.44 f N_1 \Phi_m\)可知,电源电压\(U_1\)和电源频率\(f\)不变时,\(\Phi_m\)近于不变。
这就是说,变压器铁心中主磁通最大值在变压器空载或有载时,基本上是恒定的。那么,这两种状态时的磁通势也应当是近于相等的,即

\begin{equation}
    N_1 \dot{I_1} + N_2 \dot{I_2} \approx N_1 \dot{I_0}
\end{equation}

上式中\(\dot{I_0}\)很小,于是

\begin{equation}
    N_1 \dot{I_1} \approx - N_2 \dot{I_2}
\end{equation}

可以得出

\begin{equation}
    \frac{I_1}{I_2} \approx \frac{N_2}{N_1} = \frac{1}{K}
    \label{28}
\end{equation}

这就是变压器的电流变换作用,即一次、二次绕组电流之比等于它们匝数比的倒数。

\subsubsection{阻抗变换}

根据\ref{25}和\ref{28}可得出

\begin{equation}
    \frac{U_1}{I_1} = \frac{\dfrac{N_1}{N_2}U_2}{\dfrac{N_2}{N_1}I_2} = \left(\frac{N_1}{N_2}\right)^2 \frac{U_2}{I_2}
\end{equation}

于是

\begin{equation}
    \begin{aligned}
        \left|Z^{\prime}\right| & =\left(\frac{N_1}{N_2}\right)^2|Z| \\
        & =K^2|Z|
    \end{aligned}
\end{equation}

\subsection{变压器的外特性}

变压器的外特性如图\ref{变压器的外特性曲线}所示。由特性曲线\(U_2=f(I_2)\)可见,随着二次绕组电流\(I_2\)的增大,输出电压\(U_2\)的下降逐渐明显。
电压变化率

\begin{equation}
\Delta U \%=\frac{U_{20}-U_2}{U_{20}} \times 100 \%
\end{equation}

额定容量

\begin{equation}
    S_N = U_{28} I_{28}
\end{equation}

输出有功功率

\begin{equation}
    P_2 = U_{28} I_{28} \cos \varphi_2
\end{equation}

\begin{figure}[!h]
    \centering
    \includegraphics[width = .4\textwidth]{graphics/Screenshot 2025-09-13 at 01.26.07.png}
    \caption{变压器的外特性曲线}
    \label{变压器的外特性曲线}
\end{figure}

\subsection{变压器的损耗与功率}

变压器的效率

\begin{equation}
\eta=\frac{P_2}{P_1}=\frac{P_2}{P_2+\Delta P_{\mathrm{Fe}}+\Delta P_{\mathrm{Cu}}}
\end{equation}

\subsection{特殊变压器}

\subsubsection{自藕变压器}

二次绕组是一次绕组的一部分,则
\begin{equation}
    \frac{U_1}{U_2} = \frac{N_1}{N_2} = K \quad
    \frac{I_1}{I_2} = \frac{N_2}{N_1} = \frac{1}{K}
\end{equation}

\begin{figure}[!h]
    \centering
    \includegraphics[width = .25\textwidth]{graphics/Screenshot 2025-09-15 at 18.10.20.png}
    \caption{自藕变压器的电路}
    \label{自藕变压器的电路}
\end{figure}

\section{电磁铁}

电磁铁的吸力是它的主要参数之一。直流电磁铁的吸力的大小与气隙的截面积\(S_0\)及气隙中的磁感应强度\(B_0\)的平方成正比。计算吸力的基本公式为
\begin{equation}
    F = \frac{10^7}{8 \uppi} B_0^2 S_0
\end{equation}

交流电磁铁中磁场是交变的,吸力的最大值为
\begin{equation}
    F_m = \frac{10^7}{8 \uppi} B_m^2 S_0
\end{equation}
计算时只考虑吸力的平均值
\begin{equation}
    F = \frac{10^7}{16 \uppi} B_m^2 S_0
\end{equation}

\end{document}
