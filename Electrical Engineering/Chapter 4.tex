\documentclass[12pt]{article}
%Also I made it 12pt

\usepackage[fontset=macnew]{ctex}
\usepackage{physics}
\usepackage{tikz}
\usetikzlibrary{3d,calc,patterns,chains,arrows.meta, positioning}
% \usepackage{tkz-euclide}
\usepackage{amsmath}
\usepackage{upgreek}
\usepackage{amsthm}
\usepackage{amsfonts}
\input{PomLingHandoutTemplatePreamble}
%MJKD note to future self - this preamble is just the section headers + PomLing formatting, but an ordering paradox between the two files made me combine them and re-order fontspec. *shrug* In future if it needs an update, just take the PomLing formatting file and add in the section headers for handouts.

\newcommand{\rmd}{\mathrm{d}}
\newcommand{\deriv}[2]{\frac{\rmd #1}{\rmd #2}}
\newcommand{\pderiv}[2]{\frac{\partial #1}{\partial #2}}
\newcommand{\dpderiv}[2]{\dfrac{\partial #1}{\partial #2}}
\newcommand{\dderiv}[2]{\dfrac{\rmd #1}{\rmd #2}}

\title{电动机}
\author{\href{mailto:lai-wei@whu.edu.cn}{Lai Wei}}
\date{\today}

\begin{document}

\maketitle

\section{三相异步电动机的构造}

\begin{figure}[!h]
    \centering
    \includegraphics[width = .75\textwidth]{graphics/Screenshot 2025-09-15 at 18.24.13.png}
    \caption{三相异步电动机的构造}
    \label{三相异步电动机的构造}
\end{figure}

\section{三相异步电动机的工作原理}

\subsection{旋转磁场}

\subsubsection{旋转磁场的产生}

\begin{figure}[!h]
    \centering
    \includegraphics[width = .6\textwidth]{graphics/Screenshot 2025-09-15 at 18.43.21.png}
    \caption{三相对称电流}
    \label{三相对称电流}
\end{figure}

\begin{equation*}
    \begin{aligned}
        & i_1 = I_m \sin(\omega t) \\
        & i_2 = I_m \sin(\omega t - 120^{\circ}) \\
        & i_3 = I_m \sin(\omega t + 120^{\circ})
    \end{aligned}
\end{equation*}

\subsubsection{旋转磁场的极数}

如果将三相定子绕组作不同的安排,也可产生两对、三对或更多刺激对数的旋转磁场。

\subsubsection{旋转磁场的转速}

设电流的频率为\(f_1\),即电流每秒钟交变\(f_1\)次,则旋转磁场的转速为\(n_0 = 60 f_1\)。当旋转磁场具有\(p\)对极时,磁场的转速为
\begin{equation}
    n_0 = \frac{60 f_1}{p}
\end{equation}

\subsection{电动机的转动原理}

\begin{figure}[!h]
    \centering
    \includegraphics[width = .2\textwidth]{graphics/Screenshot 2025-09-15 at 18.52.30.png}
    \caption{转子转动的原理图}
    \label{转子转动的原理图}
\end{figure}

在电动势的作用下,闭合的导条中就有电流。这电流与旋转磁场相互左右。而使转子导条受到电磁力\(F\)。电磁力的方向可应用左手定则来确定。由电磁力产生电磁转矩,转子就转动起来。由图\ref{转子转动的原理图}可知,转子转动的方向和磁极旋转的方向相同。

\subsection{转差率}

用\textbf{转差率}\(s\)来表示转子转速\(n\)与磁场转速\(n_0\)相差的程度,即
\begin{equation}
    s = \frac{n_0 - n}{n_0}
    \label{3-2}
\end{equation}

当\(n=0\)时(启动初始瞬间),\(s=1\),这时转差率最大。

式\ref{3-2}也可写为
\begin{equation}
    n = (1-s)n_0
\end{equation}

\end{document}
