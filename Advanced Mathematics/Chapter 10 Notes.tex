%!TEX program = xelatex
\documentclass[12pt, a4paper]{article}

\usepackage[dvipsnames]{xcolor}

\usepackage{fancyhdr}
\usepackage{extramarks}
\usepackage{amsmath}
\usepackage{empheq}
\usepackage{amsthm}
\usepackage{amsfonts}
\usepackage{tikz}
\usepackage{tikz-3dplot}
\usepackage[plain]{algorithm}
\usepackage{algpseudocode}

\usepackage{ctex}
\usepackage{upgreek}
\usepackage{indentfirst}
\usepackage{wrapfig}
\usepackage{subfigure}
\usepackage{pgfplots}
\usepgfplotslibrary{patchplots}
\usepgfplotslibrary{colormaps}
\usepgfplotslibrary{colorbrewer}
\pgfplotsset{compat=1.18}
\usetikzlibrary{automata,positioning,shapes.geometric,arrows.meta,patterns,calc}
\numberwithin{equation}{section}
\CTEXoptions[today=old]

%
% Basic Document Settings
%

\topmargin=-0.25in
\evensidemargin=0in
\oddsidemargin=0in
\textwidth=6.5in
\textheight=9.2in
\headsep=0.25in

\linespread{1.1}

\pagestyle{fancy}
\lhead{\hmwkAuthorName}
\chead{\hmwkClass : \hmwkTitle}
\rhead{\firstxmark}
\lfoot{\lastxmark}
\cfoot{\thepage}

\renewcommand\headrulewidth{0.4pt}
\renewcommand\footrulewidth{0.4pt}

\setlength{\parindent}{2em}  % 2em代表首行缩进两个字符

%
% Create Problem Sections
%

\newcommand{\enterProblemHeader}[1]{
    \nobreak\extramarks{}{Problem \arabic{#1} continued on next page\ldots}\nobreak{}
    \nobreak\extramarks{Problem \arabic{#1} (continued)}{Problem \arabic{#1} continued on next page\ldots}\nobreak{}
}

\newcommand{\exitProblemHeader}[1]{
    \nobreak\extramarks{Problem \arabic{#1} (continued)}{Problem \arabic{#1} continued on next page\ldots}\nobreak{}
    \stepcounter{#1}
    \nobreak\extramarks{Problem \arabic{#1}}{}\nobreak{}
}

% \setcounter{secnumdepth}{0}
\newcounter{partCounter}
\newcounter{homeworkProblemCounter}
\setcounter{homeworkProblemCounter}{0}
% \nobreak\extramarks{Problem \arabic{homeworkProblemCounter}}{}\nobreak{}

%
% Homework Problem Environment
%
% This environment takes an optional argument. When given, it will adjust the
% problem counter. This is useful for when the problems given for your
% assignment aren't sequential. See the last 3 problems of this template for an
% example.
%
\newenvironment{homeworkProblem}[1][-1]{
    \ifnum#1>0
        \setcounter{homeworkProblemCounter}{#1}
    \fi
    \section{Problem \arabic{homeworkProblemCounter}}
    \setcounter{partCounter}{1}
    \enterProblemHeader{homeworkProblemCounter}
}{
    \exitProblemHeader{homeworkProblemCounter}
}

%
% Homework Details
%   - Title
%   - Due date
%   - Class
%   - Section/Time
%   - Instructor
%   - Author
%

\newcommand{\hmwkTitle}{Multiple Integral}
\newcommand{\hmwkClass}{Advanced Mathematics}
\newcommand{\hmwkClassTime}{}
\newcommand{\myUniversiy}{Wuhan University}
\newcommand{\hmwkAuthorName}{\textbf{Lai Wei}}

%
% Title Page
%

\title{
    \vspace{2in}
    \textmd{\textbf{\hmwkClass:\ \hmwkTitle}}\\
    \vspace{0.4in}
    \large{\textit{\myUniversiy}}
    \vspace{3in}
}

\author{\hmwkAuthorName}
\date{\today}

\renewcommand{\part}[1]{\textbf{\large Part \Alph{partCounter}}\stepcounter{partCounter}\\}

%
% Various Helper Commands
%

% Useful for algorithms
\newcommand{\alg}[1]{\textsc{\bfseries \footnotesize #1}}

% % For derivatives
% \newcommand{\deriv}[1]{\frac{\mathrm{d}}{\mathrm{d}x} (#1)}

% For partial derivatives
% \newcommand{\pderiv}[2]{\frac{\partial}{\partial #1} (#2)}

% Integral dx
\newcommand{\dx}{\mathrm{d}x}

% Alias for the Solution section header
\newcommand{\solution}{\textbf{\large Solution}}

% Probability commands: Expectation, Variance, Covariance, Bias
\newcommand{\E}{\mathrm{E}}
\newcommand{\Var}{\mathrm{Var}}
\newcommand{\Cov}{\mathrm{Cov}}
\newcommand{\Bias}{\mathrm{Bias}}

% 我的newcommand
\newcommand{\degree}{^{\circ}}
\newcommand{\arrow}{-{Stealth[length=4mm,width=2mm]}}
\newcommand{\rmd}{\mathrm{d}}
\newcommand{\deriv}[2]{\frac{\rmd #1}{\rmd #2}}
\renewcommand{\parallel}{\mathrel{/\mskip-2.5mu/}}
\newcommand{\pderiv}[2]{\frac{\partial #1}{\partial #2}}
\newcommand{\parallelogram}{
	\mathord
    {\text
        {
			\tikz[baseline]
			\draw (0,.1ex) -- (.8em,.1ex) -- (1em,1.6ex) -- (.2em,1.6ex) -- cycle;
        }
    }
}

\begin{document}

\maketitle

\pagebreak

% 设置页码格式是罗马数字
\pagenumbering{roman}

% 生成目录
\tableofcontents

\pagebreak

% 设置页码格式是阿拉伯数字
\pagenumbering{arabic}

\pagebreak

\section{二重积分的概念与性质}

\subsection{二重积分的概念}

\subsubsection{定义}

    设\(f\left(x,y\right)\)是有界闭区域\(D \)上的有界函数。将闭区域\(D \)任意分成\(n \)个小闭区域
    
    $$
        \Delta \sigma_1, \Delta \sigma_2, \cdots, \Delta \sigma_n
    $$

    其中\(\Delta \sigma_i \)表示第\(i \)个小闭区域,也表示它的面积。
    在每个\(\Delta \sigma_i\)上任取一点\(\left(\xi_i, \eta_i\right)\),
    作乘积\(f\left(\xi_i, \eta_i \right)\Delta \sigma_i\)(\(i = 1, 2, \cdots n \)),
    并作和\({\displaystyle \sum_{i=1}^{n }f\left(\xi_i, \eta_i \right)\Delta \sigma_i}\)。
    如果当各小闭区域的直径中的最大值\(\lambda \rightarrow 0\)时,这和的极限总存在,
    且与闭区域\(D \)的分法及点\(\left(\xi_i, \eta_i\right)\)的取法无关,那么称此极限为函数\(f\left(x,y\right)\)
    在闭区域\(D \)上的二重积分,记作\({\displaystyle \iint_D f(x, y) \mathrm{d} \sigma}\),即

    \begin{equation}
        \iint_D f(x, y) \mathrm{d} \sigma=\lim _{x \rightarrow 0} \sum_{i=1}^n f\left(\xi_i, \eta_i\right) \Delta \sigma_i
    \end{equation}

    其中\(f\left(x,y\right)\)叫做被积函数,\(f(x, y) \mathrm{d} \sigma\)叫做被积表达式,\(\rmd \sigma\)叫做面积元素,
    \(x \)与\(y \)叫做积分变量,\(D \)叫做积分区域,\({\displaystyle \sum_{i=1}^n f\left(\xi_i, \eta_i\right) \Delta \sigma_i}\)
    叫做积分和。
    \vspace{1em}

    在二重积分的定义中对闭区域$D$的划分是任意的,
    如果在直角坐标系中用平行于坐标轴的直线网来划分$D$,
    那么除了包含边界点的一些小闭区域外,其余的小闭区域都是矩形闭区域。
    设矩形闭区域$\Delta \sigma_i$的边长为$\Delta x_j$;和$\Delta y_k$,
    则$\Delta \sigma_i=\Delta x_j \cdot \Delta y_k$.因此在直角坐标系中,
    有时也把面积元素$\mathrm{d} \sigma$记作$\mathrm{d} x \mathrm{~d} y$,而把二重积分记作

    $$
        \iint_D f(x, y) \mathrm{d} x \mathrm{~d} y,
    $$

    其中 $\mathrm{d} x \mathrm{~d} y$叫做直角坐标系中的面积元素。

\subsubsection{二重积分的几何意义}

    一般地,如果$f(x, y) \geq 0$,被积函数$f(x, y)$可以解释为曲顶柱体的顶在点$(x, y)$处的竖坐标,
    所以二重积分的几何意义就是柱体的体积。如果$f(x, y)$是负的,柱体就在$xOy$面的下方,
    二重积分的绝对值仍等于柱体的体积,但二重积分的值是负的。如果$f(x, y)$在 $D$ 的若干部分区域上是正的,
    而在其他的部分区域上是负的,那么,$f(x, y)$在$D$上的二重积分就等于$x O y$面上方的柱体体积减去$x O y$面下方的柱体体积所得之差。

\subsection{二重积分的性质}

\subsubsection{性质1}

    设\(\alpha\)和\(\beta\)为常数,则
    
    $$
        \iint_D[\alpha f(x, y)+\beta g(x, y)] \mathrm{d} \sigma=
        \alpha \iint_D f(x, y) \mathrm{d} \sigma+\beta \iint_D g(x, y) \mathrm{d} \sigma
    $$

\subsubsection{性质2}

    如果闭区域$D$被有限条曲线分为有限个部分何区域,那么在$D$的二重积分等于在各部分闭区域上的二重积分的和。

    例如$D$分为两个闭区域$D_1$与$D_2$,则
    
    $$
        \iint_D f(x, y) \mathrm{d} \sigma=\iint_{D_1} f(x, y) \mathrm{d} \sigma+\iint_{D_2} f(x, y) \mathrm{d} \sigma
    $$

    这个性质表示二重积分对于积分区域具有\textbf{可加性}。

\subsubsection{性质3}

    如果在$D$上,$f(x, y)=1$,$\sigma$为$D$的面积,那么

    $$
        \sigma=\iint_D 1 \cdot \mathrm{~d} \sigma=\iint_D \mathrm{~d} \sigma .
    $$

    这性质的几何意义是很明显的,因为高为1的平顶柱体的体积在数值上等于柱体的底面积。

\subsubsection{性质4}

    如果在$D$上,$f(x, y) \leq g(x, y)$ ,那么有
    
    $$
        \iint_D f(x, y) \mathrm{d} \sigma \leq \iint_D g(x, y) \mathrm{d} \sigma .
    $$

    特殊地,由于

    $$
    -|f(x, y)| \leq f(x, y) \leq|f(x, y)|,
    $$

    又有

    $$
        \left|\iint_D f(x, y) \mathrm{d} \sigma\right| \leq \iint_D|f(x, y)| \mathrm{d} \sigma .
    $$

\subsubsection{性质5}

    设$M$和$m$分别是$f(x, y)$在闭区域$D$上的最大值和最小值,$\sigma$是$D$的面积,则有
    
    $$
        m \sigma \leq \iint_D f(x, y) \mathrm{d} \sigma \leq M \sigma
    $$

    上述不等式是对于二重积分估值的不等式。因为$m \leq f(x, y) \leq M$,所以由性质4有

    $$
        \iint_D m \mathrm{~d} \sigma \leq \iint_D f(x, y) \mathrm{d} \sigma \leq \iint_D M \mathrm{~d} \sigma
    $$

    再应用性质1和性质3 ,便得此估值不等式。

\subsubsection{性质6}

    (二重积分的中值定理)设函数$f(x, y)$在闭区域$D$上连续,$\sigma$是$D$的面积,
    则在$D$上至少存在一点$(\xi, \eta)$,使得

    $$
        \iint_D f(x, y) \mathrm{d} \sigma=f(\xi, \eta) \sigma
    $$

\end{document}
