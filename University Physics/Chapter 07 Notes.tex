%!TEX program = xelatex
\documentclass[12pt, a4paper]{article}

\usepackage[dvipsnames]{xcolor}

\usepackage{fancyhdr}
\usepackage{extramarks}
\usepackage{amsmath}
\usepackage{amsthm}
\usepackage{amsfonts}
\usepackage{tikz}
\usepackage[plain]{algorithm}
\usepackage{algpseudocode}

\usepackage{ctex}
\usepackage{indentfirst}
\usepackage{wrapfig}
\usepackage{upgreek}
\usepackage{subfigure}
\ctexset {today=old}
\usetikzlibrary{automata,positioning,shapes.geometric,arrows.meta,patterns,calc}
\numberwithin{equation}{section}

%
% Basic Document Settings
%

\topmargin=-0.25in
\evensidemargin=0in
\oddsidemargin=0in
\textwidth=6.5in
\textheight=9.2in
\headsep=0.25in

\linespread{1.1}

\pagestyle{fancy}
\lhead{\hmwkAuthorName}
\chead{\hmwkClass : \hmwkTitle}
\rhead{\firstxmark}
\lfoot{\lastxmark}
\cfoot{\thepage}

\renewcommand\headrulewidth{0.4pt}
\renewcommand\footrulewidth{0.4pt}

\setlength{\parindent}{2em}  % 2em代表首行缩进两个字符

%
% Create Problem Sections
%

\newcommand{\enterProblemHeader}[1]{
    \nobreak\extramarks{}{Problem \arabic{#1} continued on next page\ldots}\nobreak{}
    \nobreak\extramarks{Problem \arabic{#1} (continued)}{Problem \arabic{#1} continued on next page\ldots}\nobreak{}
}

\newcommand{\exitProblemHeader}[1]{
    \nobreak\extramarks{Problem \arabic{#1} (continued)}{Problem \arabic{#1} continued on next page\ldots}\nobreak{}
    \stepcounter{#1}
    \nobreak\extramarks{Problem \arabic{#1}}{}\nobreak{}
}

% \setcounter{secnumdepth}{0}
\newcounter{partCounter}
\newcounter{homeworkProblemCounter}
\setcounter{homeworkProblemCounter}{0}
% \nobreak\extramarks{Problem \arabic{homeworkProblemCounter}}{}\nobreak{}

%
% Homework Problem Environment
%
% This environment takes an optional argument. When given, it will adjust the
% problem counter. This is useful for when the problems given for your
% assignment aren't sequential. See the last 3 problems of this template for an
% example.
%
\newenvironment{homeworkProblem}[1][-1]{
    \ifnum#1>0
        \setcounter{homeworkProblemCounter}{#1}
    \fi
    \section{Problem \arabic{homeworkProblemCounter}}
    \setcounter{partCounter}{1}
    \enterProblemHeader{homeworkProblemCounter}
}{
    \exitProblemHeader{homeworkProblemCounter}
}

%
% Homework Details
%   - Title
%   - Due date
%   - Class
%   - Section/Time
%   - Instructor
%   - Author
%

\newcommand{\hmwkTitle}{Mechanical Wave}
\newcommand{\hmwkDueDate}{\today}
\newcommand{\hmwkClass}{University Physics}
\newcommand{\hmwkClassTime}{}
\newcommand{\myUniversiy}{Wuhan University}
\newcommand{\hmwkAuthorName}{\textbf{Lai Wei}}

%
% Title Page
%

\title{
    \vspace{2in}
    \textmd{\textbf{\hmwkClass:\ \hmwkTitle}}\\
    \normalsize\vspace{0.1in}\small{Date: \hmwkDueDate}\\
    \vspace{0.1in}\large{\textit{\myUniversiy}}
    \vspace{3in}
}

\author{\hmwkAuthorName}
\date{}

\renewcommand{\part}[1]{\textbf{\large Part \Alph{partCounter}}\stepcounter{partCounter}\\}

%
% Various Helper Commands
%

% Useful for algorithms
\newcommand{\alg}[1]{\textsc{\bfseries \footnotesize #1}}

% % For derivatives
% \newcommand{\deriv}[1]{\frac{\mathrm{d}}{\mathrm{d}x} (#1)}

% For partial derivatives
\newcommand{\pderiv}[2]{\frac{\partial}{\partial #1} (#2)}

% Integral dx
\newcommand{\dx}{\mathrm{d}x}

% Alias for the Solution section header
\newcommand{\solution}{\textbf{\large Solution}}

% Probability commands: Expectation, Variance, Covariance, Bias
\newcommand{\E}{\mathrm{E}}
\newcommand{\Var}{\mathrm{Var}}
\newcommand{\Cov}{\mathrm{Cov}}
\newcommand{\Bias}{\mathrm{Bias}}

% 我的newcommand
\newcommand{\degree}{^{\circ}}
\newcommand{\arrow}{-{Stealth[length=4mm,width=2mm]}}
\newcommand{\rmd}{\mathrm{~d}}
\newcommand{\deriv}[2]{\frac{\rmd #1}{\rmd #2}}
\renewcommand{\parallel}{\mathrel{/\mskip-2.5mu/}}

\begin{document}

\maketitle

\pagebreak

% 设置页码格式是罗马数字
\pagenumbering{roman}

% 生成目录
\tableofcontents

% \newpage 
% \mbox{}
% \newpage

\pagebreak

% 设置页码格式是阿拉伯数字
\pagenumbering{arabic}

\pagebreak

    振动与波动的关系:
    振动是激发波动的波源;波动是振动的传播过程。

\section{机械波的产生和传播}

\subsection{机械波的基本概念}

\subsubsection{产生条件}

    \textbf{波}是振动状态再空间的传播过程。

    \textbf{机械波}是机械振动在弹性介质中的传播过程。

    \textbf{电磁波}是交变电场在空间的传播过程。

    \textbf{弹性介质}(elastic medium):由无穷多个质元通过弹性力结合再一起形成的连续介质。

    要产生机械波,必须满足两个条件:

    \begin{enumerate}
        \item 要有波源(即振动源);
        \item 要有能够传播机械振动的弹性介质。
    \end{enumerate}

    机械波与电磁波的共同特征:

    \begin{enumerate}
        \item 具有一定的传播速度;
        \item 能够产生干涉、衍射现象;
        \item 伴随着能量的传播;
        \item 具有相似的数学表达式。
    \end{enumerate}

\subsubsection{横波和纵波}

    \textbf{横波}:质点的振动方向与波的传播方向垂直。
    横波中凸起的位置称为波峰,下凹的位置称为波谷。

    \textbf{纵波}(又称为疏密波):质点的振动方向与波的传播方向平行。

    在机械波中,横波只能在固体中出现;纵波可在气体、液体和固体中出现。
    空气中的声波是纵波。液体表面的波动情况较复杂,不是单纯的纵波或横波。

\subsubsection{波的几何描述}

    波阵面(波面, wave surface):相位相同的点连成的面;

    波线(波射线, wave ray):波的传播方向。

    \[
        \includegraphics[scale=0.4]{Chapter 07 images/pic1.jpg}
    \]

\subsection{描述机械波的物理量}

\subsubsection{波速}

    单位时间内振动状态在介质中传播的距离称为波的传播速度,简称波速。
    在弹性介质中,机械波的传播速度取决于介质的惯性和弹性,
    也就是取决于介质的密度和弹性模量,与波源在介质中的运动速度无关。

    在拉紧的弹性绳索中,横波的传播速度为

    \begin{align}
        u = \sqrt{\frac{F }{\lambda}}
    \end{align}

    式中\(F \)为绳中的张力,\(\lambda\)是绳的质量线密度。

    在固体中,既可以传播横波,也可以传播纵波,它们的传播速度分别为

    \begin{equation}
        u=\sqrt{\frac{G}{\rho}} \quad \text{(横波)}
    \end{equation}

    \begin{equation}
        u=\sqrt{\frac{E}{\rho}} \quad(\text{(纵波)})
    \end{equation}
    
    式中\(\rho\)是固体的密度,\(G\)和\(E\)分别是固体的切变模量(shear modulus)和杨氏模量\\
    (Young's modulus),它们是反应材料形变和內应力关系的物理量,其单位都是\(\mathrm{N \cdot m^{-2}}\)。

    在液体和液体中,纵波的传播速度为

    \begin{equation}
        u=\sqrt{\frac{K}{\rho}}
    \end{equation}

    式中\(K\)为介质的体积模量,\(\rho\)是气体或液体的密度。

    对于理想气体,根据分子动理论和热力学理论,可以证明理想气体中声波的传播速度为

    \begin{equation}
        u=\sqrt{\frac{\gamma p}{\rho}}=\sqrt{\frac{\gamma R T}{M}}
        \label{7-1-5}
    \end{equation}

    式中\(M\)是气体分子的摩尔质量,是气体的摩尔定压热容与摩尔定容热容之比,
    简称摩尔热容比,\(p\)是气体的压强,\(T\)是热力学温度,\(R \)是摩尔气体常量。
    式\ref{7-1-5}表明,气体中声波的传播速度不仅与气体的性质有关,还与温度有关。

\subsubsection{波长、周期和频率}

    \textbf{波长}(wave length):波的传播方向上相邻两振动状态完全相同的质点间的距离(一完整波的长度),
    用\(\lambda\)表示。

    \textbf{周期}:波传播一个波长的距离所用的时间,用\(T \)表示。

    波速\(u \)、波长\(\lambda\)和周期\(T \)三者之间有如下关系:

    \begin{equation}
        u=\frac{\lambda}{T}
        \label{7-1-6}
    \end{equation}

    \textbf{频率}:单位时间内波向前传播的完整波的数目,是周期的倒数,用\(\nu\)表示。
    于是式\ref{7-1-6}\\可改写为

    \begin{equation}
        u=\nu \lambda
    \end{equation}

    当波源和弹性介质没有相对运动(仅仅在介质中做振动)时,波源每做一次完全振动,
    波就会向前传播一个波长的距离,这表明波源的振动周期和振动频率在数值上与波的周期和波的频率是相等的,
    即两组物理量可以通用。如果波源与介质有相对运动,则两组物理量有差异,不可通用。

\section{平面简谐波的波动表达式}

\subsection{平面简谐波的波动表达式}

    波是运动状态的传播,介质的质点并不随波传播。

    设波源在\(t\)时刻的振动方程:

    $$
        y_O=A \cos (\omega t+\varphi)
    $$

    求波线\(x \)轴上平衡位置位于\(P \)点的质点的振动表达式。
    设\(P \)点与\(O \)点的距离为\(x \),因为波动从\(O \)传到\(P \)点处所需的时间为\(\dfrac{x }{u }\),
    所以\(P \)点处的质点在\(t\)时刻的位移与\(O \)点处的质点在\(t - \dfrac{x }{u }\)时刻的位移相同,即

    $$
        y_P(t)=y_O\left(t-\frac{x}{u}\right)=A \cos \left[\omega\left(t-\frac{x}{u}\right)+\varphi_0\right]
    $$

    质点的振动表达式(即位移随时间的变化规律),忽略下标\(P \),上式可写为

    \begin{equation}
        y(x, t)=A \cos \left[\omega\left(t-\frac{x}{u}\right)+\varphi_0\right]
        \label{7-2-1}
    \end{equation}

    这就是沿\(x\)轴正方向传播的平面简谐波的波动表达式,也称为波函数。
    
    当然,我们也可以用相位的超前与落后关系求出波函数。由于沿波的传播方向,
    波线上各点的振动相位依次落后,每隔一个波长的距离,后者比前者的相位落后\(2 \uppi \)。
    因此,如果已知波长\(\lambda\),则当波从\(O \)点传到\(P\)点时,\(P \)点的振动相位比\(O\)点
    落后\(\Delta \varphi = 2\uppi \dfrac{x}{\lambda}\),即

    \begin{equation}
        \varphi_P=\varphi_0-\Delta \varphi=\varphi_0-2 \uppi \frac{x}{\lambda}
    \end{equation}

    所以\(P \)点的振动表达式,即此波的波函数为

    \begin{equation}
        y(x, t)=A \cos \left(\omega t+\varphi_P\right)=A \cos \left(\omega t-\frac{2 \uppi}{\lambda} x+\varphi_0\right)
        \label{7-2-3}
    \end{equation}

    考虑到\(u = \dfrac{\lambda }{T }\),\(\omega = \dfrac{2 \uppi}{T } = 2 \uppi \nu\),
    \(\dfrac{\omega}{u}=\dfrac{2 \uppi}{\lambda}\),所以式\ref{7-2-1}和式\ref{7-2-3}两种形式的波动表达式实际上是等价的。
    同理可得波动表达式的其他形式,如

    \begin{equation}
        y(x, t)=A \cos \left[2 \uppi\left(\frac{t}{T}-\frac{x}{\lambda}\right)+\varphi_0\right]
    \end{equation}

    如果波动沿\(x \)轴的负方向传播,那么\(P \)点处质点的振动要比\(O \)点处的质点早开始\(\dfrac{x }{u }\)时间,
    即\(P\)点处的质点在\(t\)时刻的位移与\(O\)点处的质点在\(t + \dfrac{x }{u }\)的位移相同;
    从相位的角度来看,\(P \)点处质点的振动相位要比\(O \)点的相位超前$\Delta \varphi=2 \uppi \dfrac{x}{\lambda}=\dfrac{\omega}{u} x$,
    即

    $$
        \varphi_P=\varphi_O+\Delta \varphi=\varphi_O+2 \uppi \frac{x}{\lambda}
    $$

    所以\(P \)点处质点的振动表达式为

    \begin{equation}
        y(x, t)=A \cos \left[\omega\left(t+\frac{x}{u}\right)+\varphi_0\right]
    \end{equation}
    
    或

    \begin{equation}
        y(x, t)=A \cos \left(\omega t+\frac{2 \uppi}{\lambda} x+\varphi_0\right)
    \end{equation}

    \begin{equation}
        y(x, t)=A \cos \left[2 \uppi\left(\frac{t}{T}+\frac{x}{\lambda}\right)+\varphi_0\right]
    \end{equation}

\subsection{波动表达式的物理意义}

    当\(x \)一定时,如\(x = x_0\):

    \begin{equation}
        y(t)=A \cos \left[\omega\left(t-\frac{x_0}{u}\right)+\varphi_O\right]
    \end{equation}

    是\(x = x_0\)处质点的振动表达式。

    将波动表达式分别对时间求一阶和二阶偏导数,可得波线上任一质元\(P\)振动速度和加速度分别为

    \begin{equation}
        v(x, t)=\frac{\partial y(x, t)}{\partial t}=-A \omega \sin \left[\omega\left(t \pm \frac{x}{u}\right)+\varphi_0\right]
    \end{equation}
    
    \begin{equation}
        a(x, t)=\frac{\partial^2 y(x, t)}{\partial t^2}=-A \omega^2 \cos \left[\omega\left(t \pm \frac{x}{u}\right)+\varphi_0\right]
    \end{equation}
    
    \vspace{1em}

    当\(t \)一定时,如\(t = t_0\):

    \begin{equation}
        y(x)=A \cos \left[\omega\left(t_0-\frac{x}{u}\right)+\varphi_0\right]
    \end{equation}

    表示\(t_0\)时刻\(x\)轴上各质点的位移。

\section{平面简谐波的能量}

\subsection{波的能量}

    当波动在介质中传播时,各质元中的动能和势能总是同步变化的。在波峰、
    波谷处,动能势能同时为零;在平衡位置处,两者同时到达最大值。
    进一步的定量分析还表明,任一质元中的动能和势能不仅同步变化,
    而且大小总是相等的。这就是波动的能量特点。

    以固体棒中传播的纵波为例分析波动能量的传播:

    \[
        \includegraphics[scale=0.25]{Chapter 07 images/pic2.png}
    \]

    如图所示,假设匀质细棒的橫截面积为\(S \),棒的密度为\(\rho\),
    杨氏模量为\(E \),在棒中取一段长度为\(\rmd x\)的棒元,
    其体积为\(\rmd V=S\rmd x\),质量为\(\rmd m =\rho \rmd V=\rho S\rmd x\)。
    当波动传播到这里时,这段棒元由于振动而具有动能\(\rmd E_k\)。
    由于形变而具有弹性势能\(\rmd E_p\)。 不失一般性,设细棒中平面简谐波的波函数为

    \begin{equation}
        y(x, t)=A \cos \omega\left(t-\frac{x}{u}\right)
        \label{7-3-1}
    \end{equation}

    则该棒元振动速度为

    \begin{equation}
        v=\frac{\partial y}{\partial t}=-A \omega \sin \omega\left(t-\frac{x}{u}\right)
    \end{equation}

    所以该棒元的振动动能为

    \begin{equation}
        \mathrm{d} E_{\mathrm{k}}=\frac{1}{2} \mathrm{~d} m \cdot v^2=
        \frac{1}{2} \rho \mathrm{~d} V \cdot A^2 \omega^2 \sin ^2 \omega\left(t-\frac{x}{u}\right)
    \end{equation}

    设这段横截面积为\(S \)、长为\(\rmd x\)的棒元在拉力\(F \)的作用下,其长度的伸长量为\(\rmd y\),则

    \begin{equation}
        E=\frac{F / S}{\mathrm{~d} y / \mathrm{d} x}
    \end{equation}

    不难看出:$E$的物理意义就是其单位横伐面上的拉力$F / S$(称为应力),与单位长度上的伸长量$\rmd y / \mathrm{d} x$(称为应变)之比。
    将上式改写为
    
    \begin{equation}
        F=\frac{E S}{\mathrm{~d} x} \cdot \mathrm{~d} y
    \end{equation}

    将上式与胡克定律比较,的这段棒元的劲度系数为$k=\dfrac{E S}{\mathrm{~d} x}$,所以该棒元的弹性势能为

    \begin{equation}
        \mathrm{d} E_{\mathrm{p}}=\frac{1}{2} k(\mathrm{~d} y)^2=\frac{1}{2} \frac{E S}{\mathrm{~d} x}(\mathrm{~d} y)^2
        =\frac{1}{2} E \mathrm{~d} V\left(\frac{\partial y}{\partial x}\right)^2
        \label{7-3-6}
    \end{equation}

    又因为固体中纵波的传播速度为\(u = \sqrt{\dfrac{E }{\rho}}\),所以\(E = \rho u^2\),
    再由波动表达式\ref{7-3-1},可得应变为

    \begin{equation}
        \frac{\partial y}{\partial x}=-A \frac{\omega}{u} \sin \omega\left(t-\frac{x}{u}\right)
    \end{equation}

    将上式代入式\ref{7-3-6},整理后可得该棒元的弹性势能为

    \begin{equation}
        \mathrm{d} E_{\mathrm{p}}=\frac{1}{2} \rho A^2 \omega^2 \mathrm{~d} V \sin ^2 \omega\left(t-\frac{x}{u}\right)
    \end{equation}

    比较可得

    $$
        \mathrm{d} E_{\mathrm{k}}=\mathrm{d} E_{\mathrm{p}}=
        \frac{1}{2} \rho A^2 \omega^2 \mathrm{~d} V \sin ^2 \omega\left(t-\frac{x}{u}\right)
    $$

    即在波动传播到的任意位置,任意体积元\(\rmd V\)内的动能和势能总是相等的,且同步变化。
    所以体积元\(\rmd V\)中波动的总能量为

    \begin{equation}
        \mathrm{d} E=\mathrm{d} E_{\mathrm{k}}+\mathrm{d} E_{\mathrm{p}}=\rho A^2 \omega^2 \mathrm{~d} V \sin ^2 \omega\left(t-\frac{x}{u}\right)
        \label{7-3-9}
    \end{equation}

    \textbf{讨论}:

    \begin{enumerate}
        \item 每一体积元的动能和势能值相同,且同相位;
        \item 小体积元的机械能随时间作周期性变化。
    \end{enumerate}

\subsection{波的能量密度}

    人们将介质中单位体积内具有的波动能量称为波的能量密度(energy density of wave),记为$w$,由式\ref{7-3-9}可得

    \begin{equation}
        w=\frac{\mathrm{d} E}{\mathrm{~d} V}=\rho A^2 \omega^2 \sin ^2 \omega\left(t-\frac{x}{u}\right)
    \end{equation}

    上式说明,波的能量密度同样是时间\(t\)和空间坐标\(x\)的二元函数,且沿波的传播方向(\(x\)轴)以速度\(u\)传播。
    通常把能量密度在一个时间周期\(T\)内的平均值,称为平均能量密度,并用\(\overline{w}\)表示,即

    \begin{equation}
        \overline{w}=\frac{1}{T} \int_t^{t+T} w \mathrm{~d} t
    \end{equation}

    正弦函数:

    $$
        \begin{aligned}
            \overline{w} & =\frac{1}{T} \int_0^T w \mathrm{~d} t \\
            & =\frac{1}{T} \int_0^T \rho \omega^2 A^2 \sin ^2 \omega\left(t-\frac{x}{u}\right) \mathrm{d} t=\frac{1}{2} \rho \omega^2 A^2
        \end{aligned}
    $$

    即

    \begin{equation}
        \overline{w}=\frac{1}{2} \rho A^2 \omega^2
    \end{equation}

\subsection{能流、平均能流密度}

\subsubsection{能流(面能量)}

    单位时间内通过某面积的能量——通过该面积的能流(energy flux, \(P\))。

    对垂直于波的传播方向的面积\(S\)而言:

    \[
        \includegraphics[scale=0.25]{Chapter 07 images/pic3.png}
    \]

    能流

    \begin{equation}
        P=w u S=\rho A^2 \omega^2 u S \sin ^2 \omega\left(t-\frac{x}{u}\right)
    \end{equation}

    平均能流(average energy flux)

    \begin{equation}
        \overline{P}=\overline{w} u S
    \end{equation}

\subsubsection{波的能流密度}

    能流密度:与波传播方向垂直的单位面积上通过的能流。

    平均能流密度(average energy flux density):
    与波传播方向垂直的单位面积上通过的平均能流,也称为波的强度(波强,intensity of wave),用\(I \)表示,即

    \begin{equation}
        I=\frac{\overline{P}}{S}=\overline{w} u=\frac{1}{2} \rho u \omega^2 A^2
    \end{equation}

    由此可以证明,平面简谐波在无吸收的介质中传播时振动保持不变,
    球面波的波强与离开波源的距离的平方成反比。

\section{惠更斯原理、波的衍射、反射和折射}

\subsection{惠更斯原理}

    在波的传播过程中,所到达的每一点都可看作是发射子波的波源,
    在其后的任一时刻,这些子波的包络面就是新的波前。

    \[
        \includegraphics[scale=0.2]{Chapter 07 images/pic4.jpg}
    \]

\subsection{惠更斯原理的应用}

\subsubsection{波的衍射}

    波在传播过程中遇到障得物时,能够偏窝原来的直线传播方向,绕到障碍物后面的绕射现象,
    称为波的衍射(diffraction)。

    \begin{figure}[!htbp]
        \centering  % 居中放置
        \subfigure[缝宽\(a\)远大于波长\(\lambda\)]  % 为每个图片加上编号
        {
            \begin{minipage}[b]{.3\linewidth}
                \includegraphics[scale=0.2]{Chapter 07 images/pic5.png}
                % \caption{缝宽\(a \)远大于波长\(\lambda\)}
            \end{minipage}
        }
        \subfigure[缝宽\(a \)小于波长\(\lambda\)]
        {
            \begin{minipage}[b]{.3\linewidth}
                \includegraphics[scale=0.2]{Chapter 07 images/pic6.png}
                % \caption{缝宽\(a \)小于波长\(\lambda\)}
            \end{minipage}
        }
        % \caption{}
    \end{figure}
    
    \(\dfrac{a }{\lambda}\)越小,衍射越显著。

    波长较短的波衍射不显著,所以定向性较好。

\subsubsection{波的反射和折射}

    \textbf{反射定律}
    \vspace{1em}

    \textbf{折射定律}

    当波动从一种介质进入另一种介质时,由于在两种介质中的波速不同,
    在分界面上要发生折射现象。设在介质1中的波速是\(u_1\),在介质2中波速为\(u_2\),
    设\(MN \)为两种介质的分界面。

    \[
        \includegraphics[scale=0.25]{Chapter 07 images/pic7.jpg}
    \]

    \begin{equation}
        \frac{\sin i}{\sin r}=\frac{u_1}{u_2}=n_{21}
    \end{equation}

    式中\(n_{21}=\dfrac{u_1}{u_2}\)称为介质2对介质1的相对折射率。

\section{波的干涉}

\subsection{波的叠加原理}

    几列波相遇之后, 仍然保持它们各自原有的特征(频率、波长、振幅、振动方向等)不变,
    相遇之后并按照原来的方向继续前进,好象没有遇到过其他波一样。

    在相遇区域内任一点的振动,为各列波单独存在时在该点所引起的振动位移的矢量和。
    
\subsection{波的干涉}

    频率相同、振动方向平行、相位相同或相位差恒定的两列波相遇时,
    使某些地方振动始终加强,而使另一些地方振动始终减弱的现象,
    称为波的干涉(interference)现象。

    两相干波源振动规律:

    \[
        \includegraphics[scale=0.3]{Chapter 07 images/pic8.png}
    \]

    $$
        y_1=A_1 \cos \left(\omega t-\frac{2 \uppi r_1}{\lambda}+\varphi_1\right)
    $$

    $$
        y_2=A_2 \cos \left(\omega t-\frac{2 \uppi r_2}{\lambda}+\varphi_2\right)
    $$

    \(P \)点的合振动为

    \begin{equation}
        y=y_1+y_2=A \cos (\omega t+\varphi)
    \end{equation}

    其中

    \begin{equation}
        \tan \varphi=\dfrac{A_1 \sin \left(\varphi_1-\dfrac{2 \uppi r_1}{\lambda}\right)+
        A_2 \sin \left(\varphi_2-\dfrac{2 \uppi r_2}{\lambda}\right)}{A_1 \cos \left(\varphi_1-\dfrac{2 \uppi r_1}{\lambda}\right)+A_2 \cos \left(\varphi_2-\dfrac{2 \uppi r_2}{\lambda}\right)}
    \end{equation}

    \begin{equation}
        A=\sqrt{A_1^2+A_2^2+2 A_1 A_2 \cos \left(\varphi_2-\varphi_1-2 \uppi \frac{r_2-r_1}{\lambda}\right)}
    \end{equation}

    两个波源振动传播到\(P\)点的引起的两个振动相位差值

    \begin{equation}
        \Delta \varphi=\varphi_2-\varphi_1-2 \uppi \frac{r_2-r_1}{\lambda}
    \end{equation}

    干涉叠加加强或减弱取决于\(\Delta \varphi\)。凡是满足

    \begin{equation}
        \Delta \varphi=\varphi_2-\varphi_1-2 \uppi \frac{r_2-r_1}{\lambda}= \pm 2 k \uppi, \quad k=0,1,2, \cdots
    \end{equation}

    的空间各点,\(A = A_{max} = A_1+A_2\),这些点的合振幅最大,称为\textbf{干涉加强},或\textbf{相长干涉}。

    凡是满足

    \begin{equation}
        \Delta \varphi=\varphi_2-\varphi_1-2 \uppi \frac{r_2-r_1}{\lambda}= \pm(2 k-1) \uppi, \quad k=1,2,3, \cdots
    \end{equation}

    的空间各点,\(A = A_{min} = \left|A_1-A_2\right|\),这些点的合振幅最小,称为\textbf{干涉减弱}。
    如果\(A_1=A_2\),则\(A = A_{min} = 0\),即该点的合振幅为零,这称为\textbf{相消干涉}。

    \(P\)波强度

    \begin{equation}
        I=I_1+I_2+2 \sqrt{I_1 I_2} \cos \Delta \varphi
        \label{7-5-7}
    \end{equation}

    \begin{enumerate}
        \item 相干波源在\(P\)点波强\(\neq I_1+I_2\), 而是多出一与位置有关的干涉项;
        \item 对确定点\( P\),\(\Delta \varphi\)为固定值,波强为确定值,不随时间变化;
        \item 对不同点,\(\Delta \varphi\)不同,\(I\)也不同,有强弱分布,产生干涉现象;
        \item 当\(\Delta \varphi\)为\(\uppi\)的偶数倍时,波强达极大值(干涉极大);当\(\Delta \varphi\)为\(\uppi\)的奇数倍时,干涉极小。
    \end{enumerate}

    当\(I_1=I_2=I_0\),则式\ref{7-5-7}可以写成

    \begin{equation}
        I=2 I_0(1+\cos \Delta \varphi)=4 I_0 \cos ^2 \frac{\Delta \varphi}{2}
    \end{equation}

\section{驻波}

\subsection{驻波的形成和特点}

    两列\textbf{振幅相同}的\textbf{相干波}在同一直线上\textbf{沿相反方向传播}时由于相干叠加所产生的波称为\textbf{驻波}。

    波的干涉是特定条件下的波叠加, 驻波又是特定条件下的波的干涉。

    \[
        \includegraphics[width=\linewidth]{Chapter 07 images/pic9.png}
    \]

\subsection{驻波方程}

    设有两列波相向传播,沿\(x\)轴正向传播的波为

    $$
        y_1(x, t)=A \cos \left(\omega t-\frac{2 \uppi x}{\lambda}+\varphi_1\right)
    $$

    沿\(x\)轴负向传播的波为

    $$
        y_2(x, t)=A \cos \left(\omega t+\frac{2 \uppi x}{\lambda}+\varphi_2\right)
    $$

    则合成波(即驻波)的表达式为:

    \begin{equation}
        y_{\text{合}}(x, t)=y_1(x, t)+y_2(x, t)=
        2 A \cos \left(\frac{2 \uppi}{\lambda} x+\frac{\varphi_2-\varphi_1}{2}\right) \cos \left(\omega t+\frac{\varphi_2+\varphi_1}{2}\right)
        \label{7-6-1}
    \end{equation}

    式\ref{7-6-1}即为驻波方程。

\subsubsection{波腹和波节}

    \begin{equation}
        A_{\text{合}}=\left|2 A \cos \left(\frac{2 \uppi}{\lambda} x+\frac{\varphi_2-\varphi_1}{2}\right)\right|
    \end{equation}

    驻波上各点的振幅不同,其大小与位置\(x \)有关。
    当\({\displaystyle \left|2 A \cos \left(\frac{2 \uppi}{\lambda} x+\frac{\varphi_2-\varphi_1}{2}\right)\right|=1}\),
    合振幅达到最大值(\(A_{\text{合}} = A_1+A_2\)),这就是波腹。波腹的位置条件为

    \begin{equation}
        \frac{2 \uppi}{\lambda} x+\frac{\varphi_2-\varphi_1}{2}= \pm k \uppi, \quad k=0,1,2, \cdots
    \end{equation}

    当\({\displaystyle 2 A \cos \left(\frac{2 \uppi}{\lambda} x+\frac{\varphi_2-\varphi_1}{2}\right)=0}\),
    合振幅为零(\(A_{\text{合}} = 0\)),这就是波节。波节的位置条件为

    \begin{equation}
        \frac{2 \uppi}{\lambda} x+\frac{\varphi_2-\varphi_1}{2}= \pm(2 k+1) \frac{\uppi}{2}, \quad k=0,1,2, \cdots
    \end{equation}

    所以相邻两波节或波腹之间的距离为

    \begin{equation}
        \Delta x=x_{k+1}-x_k=\frac{\lambda}{2}
    \end{equation}

\subsubsection{振动相位}

    驻波方程\ref{7-6-1}中没有\(x\)坐标,表示没有相位的传播。

    相邻两波节之间的各质点的振动相位相同;波节两侧的各质点的振动相位相反。

    驻波(已经是叠加波)不是振动相位的传播过程,驻波的波形不发生定向传播。

\subsubsection{驻波的能量}

    驻波的能量在相邻的波腹和波节间往复变化,在相邻的波节间发生动能和势能间的转换,
    动能主要集中在波腹,势能主要集中在波节,但无长距离的能量传播。

\subsection{半波损失}

    入射波从波疏介质到波密介质,在两者的界面上,反射波在反射点出现半波损失,
    引起\(\uppi\)相位的突变。

    由入射波与反射波产生驻波 与 “半波损失”。

    \[
        \includegraphics[width=\linewidth]{Chapter 07 images/pic10.png}
    \]

\end{document}
