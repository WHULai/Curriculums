\documentclass[12pt]{article}
%Also I made it 12pt

\usepackage[fontset=macnew]{ctex}
\usepackage{physics}
\usepackage{tikz}
\usetikzlibrary{3d,calc,patterns}
% \usepackage{tkz-euclide}
\usepackage{amsmath}
\usepackage{upgreek}
\usepackage{amsthm}
\usepackage{amsfonts}
% \usepackage{subfigure}
\usepackage{subcaption}
\input{PomLingHandoutTemplatePreamble}
%MJKD note to future self - this preamble is just the section headers + PomLing formatting, but an ordering paradox between the two files made me combine them and re-order fontspec. *shrug* In future if it needs an update, just take the PomLing formatting file and add in the section headers for handouts.

\newcommand{\rmd}{\mathrm{d}}
\newcommand{\deriv}[2]{\frac{\rmd #1}{\rmd #2}}
\newcommand{\pderiv}[2]{\frac{\partial #1}{\partial #2}}
\newcommand{\dpderiv}[2]{\dfrac{\partial #1}{\partial #2}}
\newcommand{\dderiv}[2]{\dfrac{\rmd #1}{\rmd #2}}

\title{电磁感应}
\author{\href{mailto:lai-wei@whu.edu.cn}{Lai Wei}}
\date{\today}

\begin{document}

\maketitle
 
\section{电磁感应的基本定律}

\subsection{电磁感应现象}

当通过导体回路的磁通量随时间发生变化时,回路中就有感应电动势产生,从而产生感应电流。这个磁通量的变化可以是由磁场变化引起的,也可以是由于导体在磁场中运动或导体回路中的一部分切割磁力线的运动而产生的。

\subsection{楞次定律}

1833年,楞次(Lenz)在进一步概括了大量实验结果的基础上,得出了确定感应电流方向的法则,称为楞次定律。这就是:闭合回路中产生的感应电流具有确定的方向,它总是使感应电流所产生的通过回路面积的磁通量,去补偿或者反抗引起感应电流的磁通量的变化。

\begin{figure}[!h]
	\centering
	\begin{subfigure}{0.25\linewidth}
		\centering
		\includegraphics[width=0.9\linewidth]{graphics/楞次定律1.png}
		\caption{磁棒插入导体圆环时}
	\end{subfigure}
	\centering
	\begin{subfigure}{0.25\linewidth}
		\centering
		\includegraphics[width=0.9\linewidth]{graphics/楞次定律2.png}
		\caption{磁棒插入导体圆环时}
	\end{subfigure}
    \caption{楞次定律的应用}
\end{figure}

\subsection{法拉第电磁感应定律}

当通过导体回路的磁通量随时间发生变化时,回路中就有感应电动势产生,从而产生感应电流。这个磁通量的变化可以是由磁场变化引起的,也可以是由于导体在磁场中运动或导体回路中的一部分切割磁力线的运动而产生的。

通过回路所包围面积的磁通量发生变化时,回路中产生的感应电动势 与磁通量对时间的变化率成正比。如果采用国际单位制,则此定律可表示为
\begin{equation}
    \mathcal{E} = - \deriv{\Phi}{t}
\end{equation}
式中的负号时楞次定律的数学表现,或者说我们可以利用式中的负号来确定回路中电磁感应电动势的方向。

如果回路是由\(N\)匝导线串联而成,那么在磁通量变化时,每匝中都将产生感应电动势。如果每匝中通过的磁通量都是相同的,则\(N\)匝线圈中的总电动势应为各匝中电动势的总和,即
\begin{equation}
    \mathcal{E}_i = - N \deriv{\varPhi}{t} = -\deriv{N \varPhi}{t} = -\deriv{\varPsi}{t}
\end{equation}
式中,\(\varPsi = N \varPhi\)称为\emph{磁链}。

如果闭合回路的电阻为\(R\),则回路中的总感应电流为
\begin{equation}
    I_i = \frac{\mathcal{E}}{R} = -\frac{N}{R} \deriv{\varPhi}{t} = -\frac{1}{R} \deriv{\varPsi}{t}
\end{equation}
利用\(I = \dderiv{q}{t}\),可得在\(t_1\)至\(t_2\)时间内通过导线上任一横截面的感应电荷量为
\begin{equation}
    q = \int_{t_1}^{t_2} I_i \rmd t = -\frac{1}{R} \int_{\varPsi_1}^{\varPsi_2} \rmd \varPsi = \frac{1}{R} \left(\varPsi_1 - \varPsi_2\right)= \frac{N}{R} \left(\varPhi_1 - \varPhi_2\right)
\end{equation}

\section{动生电动势}

\emph{动生电动势}是指:导体回路或其一部分在磁场中运动,使其回路面积或回路的法线与磁感应强度B的夹角随时间变化,从而使回路中的磁通量发生变化;

在普遍情况下,一个任意形状的导体线圈\(L\)(不一定闭合)在任意恒定的磁场中运动或发生形变时,\(\rmd L\)和\(v\)的大小和方向都可能是不同的,这时,L中的动生电动势为:
\begin{equation}
    \mathcal{E} = \int \rmd \mathcal{E} = \int_L \left(\boldsymbol{v} \times \boldsymbol{B}\right) \cdot \rmd \boldsymbol{l}
\end{equation}

\section{感生电动势}

当置于磁场中的导体回路不动,而磁场 随时间变化时,也会在导体回路中产生感应电动势,这种感应电动势称为感生电动势。

J.C.Maxwell在分析电磁感应现象的基础上,提出了一个大胆的假设:变化的磁场在其周围空间激发一种新的电场,这种电场是涡旋电场,或称感应电场。产生感生电动势的非静电力就是这个涡旋电场力。

一段导线\(ab\)上的感生电动势为
\begin{equation}
    \mathcal{E}_i = \int \rmd \mathcal{E}_i = \int_{a}^{b} \boldsymbol{E}_i \cdot \rmd \boldsymbol{l}
\end{equation}
对于一个闭合导体回路\(L\),回路内的感生电动势为
\begin{equation}
    \mathcal{E}_i = -\deriv{\varPhi}{t} = \oint_L \boldsymbol{E}_i \cdot \rmd \boldsymbol{l}
    \label{14-7}
\end{equation}
式中\(\varPhi\)是通过闭合回路\(L\)所围成曲面的磁通量。

由于磁通量\(\varPhi = \int_S \boldsymbol{B} \cdot \rmd \boldsymbol{S}\),同时考虑到回路\(L\)及其所围的曲面静止不动,即不随时间变化,代入式\ref{14-7},可得
\begin{equation}
    \oint_L \boldsymbol{E}_i \cdot \rmd \boldsymbol{l} = -\int_{S} \pderiv{\boldsymbol{B}}{t} \cdot \rmd S
    \label{14-8}
\end{equation}
式中\(S\)是以闭合回路\(L\)为周界的任意曲面,曲面\(S\)的法线正方向与回路\(L\)的绕行方向成右手螺旋关系。

静电场的电场线一般起始于正电荷、中指于负电荷,不可能形成闭合曲线。

感生电场是由变化的磁场激发的,与静止电荷无关,它的性质由式\ref{14-8}和下式给出
\begin{equation}
    \oint_S \boldsymbol{E}_i \cdot \rmd \boldsymbol{S} = 0
\end{equation}
即\emph{感生电场是无源的非保守力场},所以感生电场的电场线都是闭合曲线,所以感生电场又称为涡旋电场。

感生电场与静电场唯一的共同点就是对电荷的作用规律相同,即电荷受到的两种电场力钧可以用\(\boldsymbol{F} = q \boldsymbol{E}\)表示。

\section{自感与互感}

\subsection{自感电动势 \quad 自感}

磁链
\begin{equation}
    \varPsi = N \varPhi
\end{equation}

回路通有电流时,该电流产生的磁场会通过回路本身,对于一个确定的线圈,当其通有的电流发生变化时,穿过线圈自身所围面积的磁通量(或磁链)随之发生变化,这时线圈中产生感应电动势的现象称为自感现象。相应的感应电动势称为自感电动势线圈的形状、周围介质(非铁磁性介质)一定时,由毕奥-萨伐尔定律可知,线圈中的电流\(I\)所激发的磁感应强度与\(I\)成正比,\(B \propto I\),故通过线圈自身的磁链更\(\varPsi\)与\(I\)成正比,有
\begin{equation}
    \varPsi = L I
\end{equation}
式中的比例系数L称为自感系数,简称自感,用以描述自感现象的强弱。

回路的自感定义为,回路中的电流为单位值时通过该回路所围面积的磁链数。

\(L\)的数值与线圈的大小、形状、匝数以及其中磁介质的性质有关,即取决于线圈的性质而与线圈中的电流无关。

由法拉第电磁感应定律,线圈中产生的自感电动势为
\begin{equation}
    \mathcal{E}_L = -\deriv{\varPsi}{t} = - L \deriv{I}{t}
\end{equation}

故\(L\)又可定义为:\(L\)的大小等于当电流随时间的变化率为一单位时,回路中产生的自感电动势
\begin{equation}
    L = \frac{\mathcal{E}_L}{-\dderiv{I}{t}}
\end{equation}

\subsection{自感的计算}

对于\(N\)匝线圈组成的回路,若通过每匝线圈的磁感通量都是\(\varPhi\),即每条磁感线交链的电流是每匝中电流的\(N\)倍,那么对整个回路的磁通匝链数为:
\begin{equation}
    \varPsi = N \varPhi
\end{equation}

\begin{figure}[!h]
    \centering
    \includegraphics[width = .3\textwidth]{graphics/自感系数的计算.png}
\end{figure}

例如对于理想路线管,横截面积为\(S\),长为\(l\),总匝数\(N\),\(n = \dfrac{N}{l}\),则
\begin{equation}
    \varPsi = N \varPhi = NBS = \mu_0 n^2 I l S
\end{equation}
于是
\begin{equation}
    L = \mu_0 n^2 l S = \mu_0 \frac{N^2}{l^2} V
\end{equation}

\subsubsection{例题}

计算总匝数为N、截面为长方形的螺绕管的自感。

\begin{figure}[!h]
    \centering
    \includegraphics[width = .25\textwidth]{graphics/自感例题.png}
\end{figure}

螺绕管外部磁场为零,内部磁场为
\begin{equation}
    B = \frac{\mu_0 NI}{2\uppi S}
\end{equation}
穿过单匝线圈的磁通量
\begin{equation}
    \varPhi = h \int_{a}^{b} \frac{\mu_0 NI}{2\uppi S} \rmd S = \frac{\mu_0 NIh}{2\uppi} \ln \frac{b}{a}
\end{equation}
自感为
\begin{equation}
    L = \frac{\varPsi}{I} = \frac{\mu_0 N^2 h}{2 \uppi} \ln \frac{b}{a} = \left(2 \times 10^{-7}\right) N^2 h \ln \frac{b}{a}
\end{equation}

\subsection{互感电动势 \quad 互感}

两个相邻导体回路中分别通有电流,当任一载流线圈中的电流发生变化时,周围的磁场随之变化,通过另一个回路所围面积的磁通量也随之变化,因而会在其中产生感应电动势。这种由于一个回路中电流的变化导致在另一个回路中产生感应电动势的现象称为互感现象。相应的感应电动势称为互感电动势。

第一个线圈称为称为初级(主)线圈,第二个称为次级(副)线圈。

\begin{figure}[!h]
    \centering
    \includegraphics[width = .2\textwidth]{graphics/互感.png}
\end{figure}

线圈1的电流为\(I_1\),它激发的磁场通过线圈2的磁链为\(\varPsi_{21}\),由毕奥-萨伐尔定律知\(I_1\)与\(\varPsi_{21}\)成正比,有
\begin{equation}
    \varPsi_{21} = M_{21} I_1
\end{equation}

线圈1中电流变化时,在线圈2中激发感应电动势为
\begin{equation}
    \mathcal{E}_{21} = - \deriv{\varPsi{21}}{t} = -M_{21} \deriv{I_1}{t}
\end{equation}

\subsection{互感的计算}

\(I_1\)在1中产生的磁场为
\begin{equation}
    B_1 = \mu n_1 I_1 = \mu \frac{N_1}{l} I_1
\end{equation}
由\(I_1\)产生并通过线圈2的磁链为
\begin{equation}
    \varPsi = N_2 B_1 S = \mu \frac{N_1 N_2}{l} S I_1
\end{equation}

互感系数
\begin{equation}
    M = \frac{\varPsi_{21}}{I_1} = \mu \frac{N_1 N_2}{l} S = \mu n_1 n_2 V
\end{equation}

\begin{equation}
    L_1 = \mu n_1^2 V
\end{equation}

\begin{equation}
    L_1 = \mu n_2^2 V
\end{equation}

\begin{equation}
    M = \sqrt{L_1 L_2}
\end{equation}

一般情况下,由于有存在漏磁
\begin{equation}
    M < \sqrt{L_1 L_2} \text{或} M = k \sqrt{L_1 L_2}
\end{equation}
式中\(k\)为耦合系数,且\(0 \leq k \leq 1\)

\subsection{互感系数的对称性}

\begin{equation*}
    \begin{aligned}
        & \because \overrightarrow{B}_1 = \nabla \times \overrightarrow{A}_1 \\
        & \therefore \varPsi_{21}=\iint_{S_2} \overrightarrow{B}_1 \cdot \rmd \overrightarrow{S}_2=\iint_{S_2}\left(\nabla \times \overrightarrow{A}_1\right) \cdot \rmd \overrightarrow{S}_2=\oint_{C_2} \overrightarrow{A}_1 \cdot \rmd \overrightarrow{l}_2 \\
        & \therefore \overrightarrow{A}_1=\frac{\mu_0}{4 \uppi} \oint_{C_1} \frac{I_1 \rmd \overrightarrow{l}_1}{r_{12}} \\
        & \therefore \varPsi_{21}=\frac{\mu_0 I_1}{4 \uppi} \oint_{C_2} \oint_{C_1} \frac{\rmd \overrightarrow{l}_1 \cdot \rmd \overrightarrow{l}_2}{r_{12}} \\
        & \therefore M_{21}=\frac{\Psi_{21}}{I_1}=\frac{\mu_0}{4 \uppi} \oint_{C_2} \oint_{C_1} \frac{\rmd \vec{l}_1 \cdot \rmd \vec{l}_2}{r_{12}}
    \end{aligned}
\end{equation*}

下标关于1、2交换对称,故有\(M_{21} = N_{12}\)。

\subsubsection{例题}

密绕螺绕管总匝数为\(N\),截面为长方形,几何尺寸如图所示。试求螺绕管与其对称轴上无限长载流直导线的互感系数\(M\)。

\begin{figure}[!h]
    \centering
    \includegraphics[width = .25\textwidth]{graphics/互感例题.png}
\end{figure}

直导线激发的磁场为
\[
\overrightarrow{B} = \frac{\mu_0 I}{2 \uppi S} \hat{\varPhi}
\]

于是
\begin{equation*}
    \begin{aligned}
        \varPsi &= N \iint_S \overrightarrow{B} \cdot \rmd \overrightarrow{S} \\
        &= \frac{\mu_0 NI}{2 \uppi} \int_{a}^{b} \frac{h \rmd S}{S} \\
        &= \frac{\mu_0 NIh}{2 \uppi} \ln \frac{b}{a}
    \end{aligned}
\end{equation*}
所以
\begin{equation*}
    M = \frac{\varPsi}{I} = \frac{\mu_0 NH}{2 \uppi} \ln \frac{b}{a}
\end{equation*}

\end{document}