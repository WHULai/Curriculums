\documentclass[12pt]{article}
%Also I made it 12pt

\usepackage[fontset=macnew]{ctex}
\usepackage{physics}
\usepackage{tikz}
\usetikzlibrary{3d,calc,patterns}
% \usepackage{tkz-euclide}
\usepackage{amsmath}
\usepackage{upgreek}
\usepackage{amsthm}
\usepackage{amsfonts}
% \usepackage{subfigure}
\usepackage{subcaption}
\input{PomLingHandoutTemplatePreamble}
%MJKD note to future self - this preamble is just the section headers + PomLing formatting, but an ordering paradox between the two files made me combine them and re-order fontspec. *shrug* In future if it needs an update, just take the PomLing formatting file and add in the section headers for handouts.

\newcommand{\rmd}{\mathrm{d}}
\newcommand{\deriv}[2]{\frac{\rmd #1}{\rmd #2}}
\newcommand{\pderiv}[2]{\frac{\partial #1}{\partial #2}}
\newcommand{\dpderiv}[2]{\dfrac{\partial #1}{\partial #2}}
\newcommand{\dderiv}[2]{\dfrac{\rmd #1}{\rmd #2}}

\title{电磁感应}
\author{\href{mailto:lai-wei@whu.edu.cn}{Lai Wei}}
\date{\today}

\begin{document}

\maketitle
 
\section{电磁感应的基本定律}

\subsection{电磁感应现象}

当通过导体回路的磁通量随时间发生变化时,回路中就有感应电动势产生,从而产生感应电流。这个磁通量的变化可以是由磁场变化引起的,也可以是由于导体在磁场中运动或导体回路中的一部分切割磁力线的运动而产生的。

\subsection{楞次定律}

1833年,楞次(Lenz)在进一步概括了大量实验结果的基础上,得出了确定感应电流方向的法则,称为楞次定律。这就是:闭合回路中产生的感应电流具有确定的方向,它总是使感应电流所产生的通过回路面积的磁通量,去补偿或者反抗引起感应电流的磁通量的变化。

\begin{figure}[!h]
	\centering
	\begin{subfigure}{0.25\linewidth}
		\centering
		\includegraphics[width=0.9\linewidth]{graphics/楞次定律1.png}
		\caption{磁棒插入导体圆环时}
	\end{subfigure}
	\centering
	\begin{subfigure}{0.25\linewidth}
		\centering
		\includegraphics[width=0.9\linewidth]{graphics/楞次定律2.png}
		\caption{磁棒插入导体圆环时}
	\end{subfigure}
    \caption{楞次定律的应用}
\end{figure}

\subsection{法拉第电磁感应定律}

当通过导体回路的磁通量随时间发生变化时,回路中就有感应电动势产生,从而产生感应电流。这个磁通量的变化可以是由磁场变化引起的,也可以是由于导体在磁场中运动或导体回路中的一部分切割磁力线的运动而产生的。

通过回路所包围面积的磁通量发生变化时,回路中产生的感应电动势 与磁通量对时间的变化率成正比。如果采用国际单位制,则此定律可表示为
\begin{equation}
    \mathcal{E} = - \deriv{\Phi}{t}
\end{equation}
式中的负号时楞次定律的数学表现,或者说我们可以利用式中的负号来确定回路中电磁感应电动势的方向。

如果回路是由\(N\)匝导线串联而成,那么在磁通量变化时,每匝中都将产生感应电动势。如果每匝中通过的磁通量都是相同的,则\(N\)匝线圈中的总电动势应为各匝中电动势的总和,即
\begin{equation}
    \mathcal{E}_i = - N \deriv{\varPhi}{t} = -\deriv{N \varPhi}{t} = -\deriv{\varPsi}{t}
\end{equation}
式中,\(\varPsi = N \varPhi\)称为\emph{磁链}。

如果闭合回路的电阻为\(R\),则回路中的总感应电流为
\begin{equation}
    I_i = \frac{\mathcal{E}}{R} = -\frac{N}{R} \deriv{\varPhi}{t} = -\frac{1}{R} \deriv{\varPsi}{t}
\end{equation}
利用\(I = \dderiv{q}{t}\),可得在\(t_1\)至\(t_2\)时间内通过导线上任一横截面的感应电荷量为
\begin{equation}
    q = \int_{t_1}^{t_2} I_i \rmd t = -\frac{1}{R} \int_{\varPsi_1}^{\varPsi_2} \rmd \varPsi = \frac{1}{R} \left(\varPsi_1 - \varPsi_2\right)= \frac{N}{R} \left(\varPhi_1 - \varPhi_2\right)
\end{equation}

\section{动生电动势}

\emph{动生电动势}是指:导体回路或其一部分在磁场中运动,使其回路面积或回路的法线与磁感应强度B的夹角随时间变化,从而使回路中的磁通量发生变化;

在普遍情况下,一个任意形状的导体线圈\(L\)(不一定闭合)在任意恒定的磁场中运动或发生形变时,\(\rmd L\)和\(v\)的大小和方向都可能是不同的,这时,L中的动生电动势为:
\begin{equation}
    \mathcal{E} = \int \rmd \mathcal{E} = \int_L \left(\boldsymbol{v} \times \boldsymbol{B}\right) \cdot \rmd \boldsymbol{l}
\end{equation}

\section{感生电动势}

当置于磁场中的导体回路不动,而磁场 随时间变化时,也会在导体回路中产生感应电动势,这种感应电动势称为感生电动势。

J.C.Maxwell在分析电磁感应现象的基础上,提出了一个大胆的假设:变化的磁场在其周围空间激发一种新的电场,这种电场是涡旋电场,或称感应电场。产生感生电动势的非静电力就是这个涡旋电场力。

一段导线\(ab\)上的感生电动势为
\begin{equation}
    \mathcal{E}_i = \int \rmd \mathcal{E}_i = \int_{a}^{b} \boldsymbol{E}_i \cdot \rmd \boldsymbol{l}
\end{equation}
对于一个闭合导体回路\(L\),回路内的感生电动势为
\begin{equation}
    \mathcal{E}_i = -\deriv{\varPhi}{t} = \oint_L \boldsymbol{E}_i \cdot \rmd \boldsymbol{l}
    \label{14-7}
\end{equation}
式中\(\varPhi\)是通过闭合回路\(L\)所围成曲面的磁通量。

由于磁通量\(\varPhi = \int_S \boldsymbol{B} \cdot \rmd \boldsymbol{S}\),同时考虑到回路\(L\)及其所围的曲面静止不动,即不随时间变化,代入式\ref{14-7},可得
\begin{equation}
    \oint_L \boldsymbol{E}_i \cdot \rmd \boldsymbol{l} = -\int_{S} \pderiv{\boldsymbol{B}}{t} \cdot \rmd S
    \label{14-8}
\end{equation}
式中\(S\)是以闭合回路\(L\)为周界的任意曲面,曲面\(S\)的法线正方向与回路\(L\)的绕行方向成右手螺旋关系。

静电场的电场线一般起始于正电荷、中指于负电荷,不可能形成闭合曲线。

感生电场是由变化的磁场激发的,与静止电荷无关,它的性质由式\ref{14-8}和下式给出
\begin{equation}
    \oint_S \boldsymbol{E}_i \cdot \rmd \boldsymbol{S} = 0
\end{equation}
即\emph{感生电场是无源的非保守力场},所以感生电场的电场线都是闭合曲线,所以感生电场又称为涡旋电场。

感生电场与静电场唯一的共同点就是对电荷的作用规律相同,即电荷受到的两种电场力钧可以用\(\boldsymbol{F} = q \boldsymbol{E}\)表示。

\end{document}