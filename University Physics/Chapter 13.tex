\documentclass[12pt]{article}
%Also I made it 12pt

\usepackage[fontset=macnew]{ctex}
\usepackage{physics}
\usepackage{tikz}
\usetikzlibrary{3d,calc,patterns}
% \usepackage{tkz-euclide}
\usepackage{amsmath}
\usepackage{upgreek}
\usepackage{amsthm}
\usepackage{amsfonts}
\input{PomLingHandoutTemplatePreamble}
%MJKD note to future self - this preamble is just the section headers + PomLing formatting, but an ordering paradox between the two files made me combine them and re-order fontspec. *shrug* In future if it needs an update, just take the PomLing formatting file and add in the section headers for handouts.

\newcommand{\rmd}{\mathrm{d}}
\newcommand{\deriv}[2]{\frac{\rmd #1}{\rmd #2}}
\newcommand{\pderiv}[2]{\frac{\partial #1}{\partial #2}}
\newcommand{\dpderiv}[2]{\dfrac{\partial #1}{\partial #2}}
\newcommand{\dderiv}[2]{\dfrac{\rmd #1}{\rmd #2}}

\title{磁介质}
\author{\href{mailto:lai-wei@whu.edu.cn}{Lai Wei}}
\date{\today}

\begin{document}

\maketitle
 
\section{磁介质的磁化}

\subsection{磁介质的磁化 \quad 相对磁导率}

被磁化后的磁介质会激发附加磁场\(\boldsymbol{B^\prime}\),从而影响介质内外的磁场分布。磁介质内的磁感应强度为
\begin{equation}
    \boldsymbol{B} = \boldsymbol{B_0} + \boldsymbol{B^ \prime}
\end{equation}

相对磁导率
\begin{equation}
    \mu_r = \frac{B}{B_0}
\end{equation}

相对磁导率\(\mu_r\)的大小只与磁介质的性质有关,根据\(\mu_r\)的大小将磁介质分成顺磁质、抗磁质和铁磁质三类。

\begin{enumerate}
    \item 顺磁质\(\mu_r\)略大于1,即\(\mu_r \approx 1 + 10^{-4} > 1\)
    \item 抗磁质\(\mu_r\)略大于1,即\(\mu_r \approx 1 - 10^{-5} < 1\)
    \item 铁磁质\(\mu_r >> 1\),磁化后产生的磁场\(\boldsymbol{B^\prime}\)的方向与原磁场\(\boldsymbol{B_0}\)的方向相同,且\(\boldsymbol{B^\prime} >> \boldsymbol{B_0}\)
\end{enumerate}

\subsection{磁化强度和磁化电流}

将磁介质中单位体积内分子磁矩的矢量和称为\emph{磁化强度},并用\(\boldsymbol{M}\)来表示,即
\begin{equation}
    \boldsymbol{M} = \frac{\sum \boldsymbol{m_{\text{分子}}}}{\Delta V} = \frac{\sum \boldsymbol{m} + \sum \Delta \boldsymbol{m}}{\Delta V}
\end{equation}
在国际单位制中,\(\boldsymbol{M}\)的单位为\(\mathrm{A} \cdot \mathrm{m}^{-1}\)。

设有一“无限长”的载流直螺线管,管内充满均匀磁介质,电流在螺线管内激发匀强磁场。在此磁场中磁介质被均匀磁化,这时磁介质中各个分子电流平面将转到与磁场的方向相垂直。对磁介质的整体来说,未被抵消的分子电流是沿着柱面流动的,称为安培表面电流(或叫磁化面电流),用\(I_s\)表示。沿轴线单位长度上的磁化电流称为磁化电流密度,用\(\boldsymbol{j_s}\)表示。于是,沿棒的轴线方向长为\(L\)的表面上,磁化电流的大小为\(I_s = j_s L\)。介质被磁化的程度越大,介质表面的磁化电流越大。

\begin{figure}[!h]
    \centering
    \includegraphics[width = .4\textwidth]{graphics/磁化电流.png}
\end{figure}

对顺磁性物质,安培表面电流和螺线管上导线中的电流方向相同;对抗磁性物质,则两者方向相反。

即当磁化强度的方向与介质表面平行时,磁介质表面磁化电流密度的大小等于该处磁化强度的大小,即
\begin{equation}
    M = \frac{\sum  \boldsymbol{m}_{\text{分子}}}{\Delta V} = \frac{j_s L S}{L S} = j_s
\end{equation}

如果磁化强度的方向与介质表面不平行,可以证明
\begin{equation}
    \boldsymbol{j}_s = \boldsymbol{M} \times \boldsymbol{e}_n
\end{equation}
式中\(\boldsymbol{e}_n\)为磁介质表面的外法线单位矢量。该式表明,磁化强度\(\boldsymbol{M}\)沿磁介质表面的切向分量等于介质表面的磁化电流密度。

可以证明:磁化强度\(\boldsymbol{M}\)与磁化电流\(I_s\)之间的普遍关系为
\begin{equation}
    \oint_L \boldsymbol{M} \cdot \rmd \boldsymbol{L} = \sum I_s
    \label{13-6}
\end{equation}

\section{有磁介质存在时的安培环路定理和高斯定理}

\subsection{有磁介质存在时的安培环路定理 \quad 磁场强度}

有磁介质存在时,空间任意一点的磁感应强度\(\boldsymbol{B}\)应由导线中的传导电流\(I_0\)和磁介质表面的磁环电流\(I_s\)共同产生,因此有磁介质存在时,磁场的安培环路定理应写成
\begin{equation}
    \oint_L = \boldsymbol{B} \cdot \rmd \boldsymbol{L} = \mu_0 \left(\sum I_0 + \sum I_s\right)
\end{equation}
式中,\(\sum I_0\)是穿过闭合回路\(L\)所围面积的传导电流的代数和,\(\sum I_s\)时磁化电流的代数和。

如果将磁化强度\(\boldsymbol{M}\)与磁化电流\(I_s\)的普遍关系式\ref{13-6}代入上式,则有
\begin{equation}
    \oint_L \boldsymbol{B} \cdot \rmd \boldsymbol{L} = \mu_0 \left(\sum I_0 + \oint_L \boldsymbol{M} \cdot \rmd \boldsymbol{L}\right)
\end{equation}
或写为
\begin{equation}
    \oint_L \left(\frac{\boldsymbol{B}}{\mu_0} - \boldsymbol{M}\right) \cdot \rmd \boldsymbol{L} = \sum I_0
    \label{13-9}
\end{equation}

引入一个新的辅助物理量,称为\emph{磁场强度},用\(\boldsymbol{H}\)表示,并定义
\begin{equation}
    \boldsymbol{H} = \frac{\boldsymbol{B}}{\mu_0} - \boldsymbol{M}
\end{equation}
则式\ref{13-9}可写为
\begin{equation}
    \oint_L \boldsymbol{H} \cdot \rmd \boldsymbol{L} = \sum I_0
    \label{13-11}
\end{equation}

式\ref{13-11}称为有磁介质时的安培环路定理,它表明磁场强度\(\boldsymbol{H}\)沿任一闭合回路\(L\)的线积分(或环流)等于通过该回路\(L\)所围成面积的传导电流的代数和。

\subsection{磁介质的磁化特性}

在各向同性的磁介质内部,任意一点的磁化强度\(\boldsymbol{M}\)和磁场强度\(\boldsymbol{H}\)成正比,即
\begin{equation}
    \boldsymbol{M} = \chi_m \boldsymbol{H}
\end{equation}
式中\(\chi_m\)为比例系数,称为介质的\emph{磁化率},对于顺磁质,\(\chi_m > 0\),表明顺磁质中\(\boldsymbol{M}\)和\(\boldsymbol{H}\)的方向相同;对于抗磁质,\(\chi_m > 0\),表明抗磁质中\(\boldsymbol{M}\)和\(\boldsymbol{H}\)的方向相反。

于是
\begin{equation}
    \boldsymbol{B} = \mu_0 (1 + \chi_m)\boldsymbol{H} = \mu_0 \mu_r \boldsymbol{H}= \mu \boldsymbol{H}
\end{equation}
式中
\begin{equation}
    \mu_r = 1 + \chi_m, \quad \mu = \mu_0\mu_r
\end{equation}
\(\mu_r\)和\(\mu_0\)分别称为磁介质的相对磁导率和绝对磁导率(有时也简称为磁导率)。

对于真空中的磁场,由于\(\boldsymbol{M} = 0\),则\(\chi_m = 0\),\(\boldsymbol{B} = \mu_0 \boldsymbol{H}\),这表明真空的相对磁导率\(\mu_r = 1\)。

\subsection{有磁介质存在时的高斯定理}

\begin{equation}
    \oint_S \boldsymbol{B} \cdot \rmd \boldsymbol{S} = 0
\end{equation}

\end{document}